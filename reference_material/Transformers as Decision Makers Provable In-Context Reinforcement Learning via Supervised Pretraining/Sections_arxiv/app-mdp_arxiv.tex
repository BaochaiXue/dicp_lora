\section{Learning in-context RL in markov decision processes}






















Throughout this section, we use $c>0$ to denote universal constants whose values may vary from line to line.
Moreover, for notational simplicity, we use $\conO(\cdot)$ to hide universal constants,  $\tcO(\cdot)$ to hide poly-logarithmic terms in $(\horizon,\Numepi,\Numst,\Numact,1/\temp)$.


This section is organized as follows. Section~\ref{sec:tf_embed_mdp} discusses the embedding and extraction formats of transformers for Markov decision processes. Section~\ref{sec:example_ucbvi} describes the UCB-VI and the soft UCB-VI algorithms.  We prove Theorem~\ref{thm:approx_ucbvi} in Section~\ref{sec:pf_approx_ucbvi} and prove Theorem~\ref{thm:ucbvi_icrl-main} in Section~\ref{sec:pf_thm:ucbvi_icrl-main}.













\subsection{Embedding and extraction mappings}\label{sec:tf_embed_mdp}

To embed MDP problems into transformers, we consider an embedding similar to that for linear bandits. For each episode $k\in[\Numepi]$, we construct $2\horizon+1$ tokens. Concretely, for each $t\in[\totlen]$ in the $k$-th episode, we write  $t=\horizon(k-1)+h$  and construct two tokens
\[
\begin{aligned}
\bh_{2(t-1)+k}=
\left[
\begin{array}{cc}
     \bzero_{\Numact+1} \\
     \hdashline
     \state_{k,h}\\
      \hdashline
    \bzero_{\Numact}\\
    \hdashline
    \bzero\\
    \posv_{2(t-1)+k}
\end{array}\right]
=:
\begin{bmatrix}
     \bh_{2(t-1)+k}^{\parta} \\  \bh_{2(t-1)+k}^{\partb}\\  \bh_{2(t-1)+k}^{\partc}\\   \bh_{2(t-1)+k}^{\partd}\\
\end{bmatrix},~~
\bh_{2t-1+k}=
\left[
\begin{array}{cc}
     \action_{k,h} \\
      \reward_{k,h}\\
      \hdashline
      \bzero_{\Numst}\\
      \hdashline
      \bzero_{\Numact}\\
     \hdashline
      \bzero\\
      \posv_{2t-1+k}
\end{array}\right]=:
\begin{bmatrix}
    \bh_{2t-1+k}^{\parta} \\  \bh_{2t-1+k}^{\partb}\\   \bh_{2t-1+k}^{\partc}\\   \bh_{2t-1+k}^{\partd}
\end{bmatrix},
\end{aligned}
\]
where  $\state_{k,h},\action_{k,h}$ are represented using one-hot embedding (we let $\state_{k,\horizon+1}=\bzero_\Numst$), $\bh^\partc_{2(t-1)+k}$ is used to store the (unnormalized)  policy at time  $t$ given current state $\state_{k,h}$, $\bzero$ in $\bh^\partd$ denotes an additional zero vector. At the end of each episode $k$, we add an empty token
$$
\bh_{(2\horizon+1)k}=\bh^{\emp}_{k}:=\begin{bmatrix}
    \bzero &\posv_{(2\horizon+1)k}
\end{bmatrix}^\top
$$ to store intermediate calculations. We also include in the tokens  the positional embedding $\posv_i:=(k,h,v_i,i,i^2,1)^\top$ for $i\in[2\totlen+\Numepi]$, where $\oddeven_i:=\bone_{\{\bh_i^\parta=\bzero\}}$  denote the tokens that do not embed actions and rewards.    In addition, we define  the token matrix $\bH_t:=\begin{bmatrix}
    \bh_1,\ldots,\bh_{2t-1+k}
\end{bmatrix}\in\R^{D\times (2t-1+k)}$ for all $t\in[\totlen]$.




\paragraph{Offline pretraining}
Similar to the bandit setting, during
pretraining the transformer $\TF_\tfpar$ takes in   $\bH_\totlen^\pre:=\bH_\totlen$ as the input token matrix, and generates $\bH_\totlen^\post:=\TF_\tfpar(\bH_\totlen^\pre)$ as the output. For each time $t\in[\totlen]$, we define the  induced policy  $\sAlg_\tfpar(\cdot|\dset_{t-1},\state_t):=\frac{\exp(\bh^{\post,\partc}_{2(t-1)+k})}{\|\exp(\bh^{\post,\partc}_{2(t-1)+k})\|_1}\in\Delta^\Numact$, whose $i$-th entry is the probability of selecting the $i$-th action (denoted by the  one-hot vector $\be_i$) given $(\dset_{t-1},\state_t)$. We then find the transformer $\esttfpar\in\tfparspace$ by solving Eq.~\eqref{eq:general_mle}.

\paragraph{Rollout}
At each time $t\in[\totlen]$, given the  current state $\state_t$ and  previous data $\dset_{t-1}$, we first construct the token matrix $\bH^{\pre}_{\roll,t}\in\R^{D\times 2(t-1)+k}$ that consists of tokens up to the first token for time $t$.   The transformer then takes $\bH^{\pre}_{\roll,t}$ as the input  and generates $\bH^{\post}_{\roll,t}=\TF_\tfpar(\bH^{\pre}_{\roll,t})$. Next,  the agent selects an action $\action_t\in\actionsp$ following  the induced  policy $\sAlg_\tfpar(\cdot|\dset_{t-1},\state_t):=\frac{\exp(\bh^{\post,\partc}_{2(t-1)+k})}{\|\exp(\bh^{\post,\partc}_{2(t-1)+k})\|_1}\in\Delta^\Numact$ and observes the reward $\reward_t$ and next state $\state_{t+1}$ ($\state_{t+1}\sim\init$ if $t$ is the last time step in an episode).





\paragraph{Embedding and extraction mappings}
To integrate the above construction into our general framework in Section~\ref{sec:framework}, for $t=(k-1)\horizon+h$,  we have the embedding vectors $$\embedmap(\state_t):=\bh_{2(t-1)+k},~~~\embedmap(\action_t,\reward_t):=\bh_{2t-1+k}.$$  For $N\geq 1$, write $$\lceil(N+1)/2\rceil=(k_N-1)\horizon+h_N$$ for some $h_N\in[\horizon]$, and define the  concatenation operator
\begin{align*}
\cat(\bh_1, \ldots, \bh_N): = [\bh_1, \ldots,\bh_{2\horizon},\bh_{1}^{\emp}, \bh_{2\horizon+1},\ldots,\bh_{4\horizon},\bh_2^\emp,\bh_{4\horizon+1},\ldots, \bh_N]\in\R^{N+k_N-1},
\end{align*}
where we insert an empty token $\bh_k^\emp$ (i.e., a token with $\bh^{\{\parta,\partb,\partc\}}=\bzero$) at the end of  each episode $k$.

In this case, we have the input token matrix  $$\bH=\bH^\pre_{\roll,t}: = \cat(\embedmap(\state_1), \embedmap(\action_1, \reward_1), \ldots, \embedmap(\action_{t-1}, \reward_{t-1}), \embedmap(\state_t)) \in \R^{D \times [2(t-1)+k]},$$ the output token matrix $\bar{\bH}=\bH^\post_{\roll,t}$, and the linear extraction map $\extractmap$  satisfies $$\extractmap\cdot\bar{\bh}_{-1}=\extractmap\cdot\bar{\bh}^\post_{2(t-1)+k}=\bh^{\post,\partc}_{2(t-1)+k}.$$






























\subsection{UCB-VI and soft UCB-VI}\label{sec:example_ucbvi}
We show that transformers with the embedding in Section~\ref{sec:tf_embed_mdp}  can approximately implement the  UCB-VI algorithm in~\cite{azar2017minimax}. Namely, UCB-VI implements the following steps:

for each episode $k\in[\Numepi]$ and each step $h=\horizon,\ldots,1$
\begin{enumerate}
    \item Compute the estimated transition matrix $\tresttransit_h(\state'|\state,\action):=   \frac{\Numvi_h(\state,\action,\state')}{\Numvi_h(\state,\action)\vee 1}$,  where $\Numvi_h(\state,\action,\state')$ denotes the number of times the state-action-next-state tuple $(\state,\action,\state')$ has been visited in the first $k-1$ episodes, and $\Numvi_h(\state,\action)=\sum_{\state'}\Numvi_h(\state,\action,\state')$ (we assume $\Numvi_\horizon(\state,\action,\state')=0$ and let $\Numvi_\horizon(\state,\action)$  be the number of times $(\state,\action)$ is visited at timestep $\horizon$).
    \item Calculate the estimated Q-function \begin{align*}
\trestQfun_h(\state,\action)=\min\{\horizon,\reward_h(\state,\action)+\bonus_h(\state,\action)+\sum_{\state'\in\statesp}\tresttransit_h(\state'\mid\state,\action)\trestVfun_{h+1}(\state')\},\end{align*}
where the bonus $\bonus_h(\state,\action)=2\horizon\sqrt{\frac{\log(\Numst\Numact\totlen/\delta)}{\Numvi_h(\state,\action)\vee1}}$,  $\trestVfun_{\horizon+1}(\state):=0$ for all $\state\in\statesp$ and $\trestVfun_h(\state):=\max_{\action\in\actionsp}\trestQfun_h(\state,\action)$.
\end{enumerate} Throughout this section, we choose the small probability $\delta=1/(\Numepi\horizon)$.


During policy execution, at each step $h\in[\horizon]$, UCB-VI takes the greedy action $\action_h:=\argmax_{\action}\trestQfun(\state_h,\action)$ and observes the reward and next state $(\reward_h,\state_{h+1})$. To facilitate pretraining, in this work we consider a soft version of UCB-VI, which takes action $\action_h$  following the softmax policy
\begin{align*}
\plc_h(\action|\state_h)=\frac{\exp(\trestQfun_h(\state_h,\action)/\temp)}{\lone{\exp(\trestQfun_h(\state_h,\action)/\temp)}}
\end{align*}
using the estimated $Q$-function for some sufficiently small $\temp>0$. Note that soft UCB-VI recovers UCB-VI as $\temp\to 0$.



















\subsection{Proof of Theorem~\ref{thm:approx_ucbvi}}\label{sec:pf_approx_ucbvi}
Throughout the proof, we abuse the notations $\bh_i^{\star}$ for $\star\in\{\parta,\partb,\partc,\partd\}$ to denote the corresponding positions in the token vector $\bh_i$. For any $t'\in[\totlen]$, we let $k(t'),h(t')$ be the non-negative integers such that $t'=\horizon(k(t')-1)+h(t')$ and $h(t')\in[\horizon]$. For the current time $t$, we use the shorthands $k=k(t),h=h(t)$.  For a token index $i\in[(2\horizon+1)\Numepi]$, let $\bar{k}(i),\bar{h}(i)$ be the episode and time step the $i$-th token corresponds to (for the empty tokens we set $h=\horizon+1$).
Given the input token matrix $\bH^\pre_{\roll,t}$, we construct a transformer that implements the following steps on the last token.
$\bh^{\star}_{2(t-1)+k}=\bh^{\pre,\star}_{2(t-1)+k}$ for $\star\in\{\parta,\partb,\partc,\partd\}$
\begin{align}
    \begin{bmatrix}
    \bh_{2(t-1)+k}^{\pre,\parta} \\  \bh_{2(t-1)+k}^{\pre,\partb}\\  \bh_{2(t-1)+k}^{\pre,\partc}\\   \bh_{2(t-1)+k}^{\pre,\partd}
\end{bmatrix}&
\xrightarrow{\text{step 1}}
   \begin{bmatrix}
    \bh_{2(t-1)+k}^{\pre,\{\parta,\partb,\partc\}} \\
        \Numvi_{1}(\state,\action,\state') \\ \vdots \\
         \Numvi_{\horizon}(\state,\action,\state')
         \\
        \Numvi_{1}(\state,\action) \\\vdots\\
        \Numvi_{\horizon}(\state,\action,\state')
         \\
         \Numvi_{1}(\state,\action)\reward_{1}(\state,\action) \\\vdots\\
        \Numvi_{\horizon}(\state,\action,\state') \reward_{\horizon}(\state,\action)
         \\
         \star\\ \bzero \\\posv_{2(t-1)+k}
\end{bmatrix}
\xrightarrow{\text{step 2}}
\begin{bmatrix}
    \bh_{2(t-1)+k}^{\pre,\{\parta,\partb,\partc\}} \\
         \esttransit_{1}(\state,\action,\state') \\ \vdots \\
         \esttransit_{\horizon}(\state,\action,\state')  \\ \star
        \\ \bzero \\\posv_{2(t-1)+k}
\end{bmatrix}
\xrightarrow{\text{step 3}}
\begin{bmatrix}
    \bh_{2(t-1)+k}^{\pre,\{\parta,\partb,\partc\}} \\
         \estQfun_{1}(\state,\action,\state') \\ \vdots \\
         \estQfun_{\horizon}(\state,\action,\state')  \\
           \estVfun_{1}(\state) \\ \vdots \\
         \estVfun_{\horizon}(\state)  \\ \star
        \\ \bzero \\\posv_{2(t-1)+k}
\end{bmatrix}\notag\\
&
\xrightarrow{\text{step 4}}
\begin{bmatrix}
    \bh_{2(t-1)+k}^{\pre,\{\parta,\partb\}}\\ \frac{\estQfun_h(\state_t,\action_1)}{\temp}\\\vdots\\
\frac{\estQfun_h(\state_t,\action_1)}{\temp}
    \\ \bh_{2(t-1)+k}^{\partd}
\end{bmatrix}
=:
\begin{bmatrix}
    \bh_{2(t-1)+k}^{\post,\parta} \\  \bh_{2(t-1)+k}^{\post,\partb}\\  \bh_{2(t-1)+k}^{\post,\partc}\\   \bh_{2(t-1)+k}^{\post,\partd}
\end{bmatrix},\label{eq:ucbvi_roadmap}
\end{align}
where $\Numvi_{\sh}(\state,\action,\state'),\esttransit_{\sh}(\state,\action,\state'),\estQfun_{\sh}(\state,\action,\state')\in\R^{\Numst^2\times\Numact},~\Numvi_{\sh}(\state,\action)\in\R^{\Numst\times\Numact},\estVfun_{\sh}(\state)\in\R^\Numst$ for all $\sh\in[\horizon]$, and $\star$ denote additional quantities in $\bh^\partd_{2(t-1)+k}$. Given the current state $\state_t$, the transformer $\TF_\tfpar(\cdot)$ generates the policy $$\sAlg_\tfpar(\cdot|\dset_{t-1},\state_t):=\frac{\exp(\bh^{\post,\partc}_{2(t-1)+k})}{\|\exp(\bh^{\post,\partc}_{2(t-1)+k})\|_1}\in\Delta^\Numact.$$
We claim the following results which we will prove later.

\begin{enumerate}[label=Step \arabic*,ref= \arabic*]
    \item\label{mdp_step1}There exists an attention-only transformer $\TF_\btheta(\cdot)$ with     $$L=4,~~\max_{\ell\in[L]}M^{(l)}\leq \conO(\horizon\Numst^2\Numact),~~~ \nrmp{\btheta}\leq \conO(\horizon\Numepi+\horizon\Numst^2\Numact) $$
 that implements step 1 in~\eqref{eq:roadmap_ts}.
   \item \label{mdp_step2}
 There exists a one-layer  transformer $\TF_\btheta(\cdot)$ with
$$
L=1,~~\head\leq \conO(\horizon\Numst^2\Numact),~~~ \hidden\leq \conO(\Numepi^2\horizon\Numst^2\Numact),~~~\nrmp{\btheta}\leq \tcO(\horizon\Numst^2\Numact+\Numepi^3+\Numepi\horizon)
$$
 that implements step 2 in~\eqref{eq:roadmap_ts}.
  \item\label{mdp_step3}  There exists a transformer  $\TF_\tfpar(\cdot)$ with
    $$L=2\horizon,~~\max_{\ell\in[L]}M^{(l)}\leq 2\Numst\Numact,~~\max_{\ell\in[L]}\hidden^{(l)}\leq 3\Numst\Numact,~~~ \nrmp{\btheta}\leq  \conO(\horizon+\Numst\Numact) $$
 that implements step 3 (i.e., value iteration) in~\eqref{eq:roadmap_ts}.
  \item \label{mdp_step4}
 There exists an attention-only transformer $\TF_\btheta(\cdot)$ with
$$\layer=3,~~\max_{\ell\in[\layer]}\head^\lth=\conO(\horizon\Numact),~~\nrmp{\btheta}\leq \conO(\horizon(\Numepi+\Numact)+{1}/{\temp}) $$
 that implements step 4 in~\eqref{eq:roadmap_ts}.
\end{enumerate}
From the construction of Step~\ref{mdp_step1}---\ref{mdp_step4}, we verify that one can choose the constructed transformer to have the embedding dimension $D=\conO(\horizon\Numst^2\Numact)$. Moreover, due to the boundedness of the reward function, $Q$-function  and the fact that the bonus $\bonus(\state,\action)\leq\tcO(\horizon)$, we verify that there exists some $\clipval>0$ with $\log \clipval=\tcO(1)$ such that $\|\bh_i^{\lth}\|_2\leq \clipval$ for all layer $\ell\in[\layer]$ and all token $i\in[\Numepi(2\horizon+1)]$. Therefore, similar to what we do in the proof of Theorem~\ref{thm:approx_smooth_linucb},~\ref{thm:approx_thompson_linear}, we may w.l.o.g. consider transformers without truncation (i.e.,  $\clipval=\infty$) in our construction of step 1---4 in~\eqref{eq:ucbvi_roadmap}.




\paragraph{Proof of Step~\ref{mdp_step1}} We prove this step by constructing a transformer that implements the following two steps:
\begin{enumerate}[label= Step 1\alph*, ref= 1\alph*]
    \item\label{mdp_step1a} For each $t'< t$ with $t'=(k'-1)\horizon+h'$ for some $h'\in[\horizon]$, we add  $\state_{k',h'},(\action_{k',h'},\reward_{k',h'})$ from  $\bh^\partb_{2(t'-1)+k'}$ and $\bh^\parta_{2t'-1+k'}$  to  $\bh^\partd_{2t'+k'}$.
    \item\label{mdp_step1b} Compute $\Numvi_{\sh}(\state,\action,\state'),\Numvi_{\sh}(\state,\action)$ for $\sh\in[\horizon]$ and assign them to the current token $\bh^\partd_{2(t-1)+k}$.
\end{enumerate}
For step~\ref{mdp_step1a}, we can construct a two-layer attention-only transformer with $\bQ^{(1)}_{1,2,3},\bK^{(1)}_{1,2,3},\bV^{(1)}_{1,2,3}$ such that for all $i\leq 2(t-1)+k$
\begin{align*}
&\bQ^{(1)}_{1}\bh^{(0)}_{i}=
\begin{bmatrix}
        \bar{k}(i)+1-\oddeven_{i}\\
        \tfthres\\
        1\\
       i
    \end{bmatrix},~~ \bK^{(1)}_{1}\bh^{(0)}_{i}=\begin{bmatrix}
        -\tfthres\\\bar{k}(i)\\ i+3\\-1
\end{bmatrix},~~ \bV^{(1)}_{1}\bh^{(0)}_{2(t'-1)+k'}=
\begin{bmatrix}
\bzero\\\bzero_{\Numact+1}\\\state_{k',h'}\\\bzero\end{bmatrix},\\
&~~~
\bV^{(1)}_{1}\bh^{(0)}_{2t'-1+k'}=
\begin{bmatrix}
\bzero\\\action_{k',h'}\\
\reward_{k',h'}
\\\bzero_\Numst\\\bzero,\end{bmatrix}
\end{align*}
where we choose $\tfthres=4$ and $\bV\bh^{(0)}$ are supported on some entries in $\bh^{(0),\partd}$. Moreover, we choose $\bQ^{(1)}_{3}=\bQ^{(1)}_{2}=\bQ^{(1)}_1$, $\bV^{(1)}_2=\bV^{(1)}_3=-\bV^{(1)}_1$ and $\bK^{(1)}_2,\bK^{(1)}_3$ such that
\begin{align*}
    \bK^{(1)}_{2}\bh^{(0)}_{i}=\begin{bmatrix}
        -\tfthres\\\bar{k}(i)\\ i+2\\-1
\end{bmatrix},~~\bK^{(1)}_{3}\bh^{(0)}_{i}=\begin{bmatrix}
        -\tfthres\\\bar{k}(i)\\ i+1\\-1.
\end{bmatrix}
\end{align*}
We verify that $\lops{\bQ^{(1)}_{\star}},\lops{\bK^{(1)}_{\star}}= 4,\lops{\bV^{(1)}_{\star}}=1$ for $\star\in[3]$.
Summing up the  heads, we obtain the following update on a subset of coordinates in $\bh^{(0),\partd}_{2t'+k'}$:
\begin{align*}
\bzero_{\Numst+\Numact+1}
    &\rightarrow
\bzero_{\Numst+\Numact+1}+\sum_{j=1}^3\sum_{i=1}^{2t'+k'}\sigma(\<\bQ^{(1)}_{j}\bh^{(0)}_{2t'+k'},\bK^{(1)}_{j}\bh^{(0)}_{i}\>)\bV_j\bh^{(0)}_i\\
    &=\frac{1}{2t'+k'} [(\bV^{(1)}_1\bh^{(0)}_{2t'+k'-2}+2\bV^{(1)}_1\bh^{(0)}_{2t'+k'-1}+3\bV^{(1)}_1\bh^{(0)}_{2t'+k'})\\
    &\qquad-(\bV^{(1)}_1\bh^{(0)}_{2t'+k'-1}+2\bV^{(1)}_1\bh^{(0)}_{2t'+k'})-\bV^{(1)}_1\bh^{(0)}_{2t'+k'})]\\
    &=\frac{1}{2t'+k'}(\bV^{(1)}_1\bh^{(0)}_{2t'+k'-2}+\bV^{(1)}_1\bh^{(0)}_{2t'+k'-1})\\
    &=\frac{1}{2t'+k'}\begin{bmatrix}
        \action_{k',h'} \\\reward_{k',h'}\\\state_{k',h'}
    \end{bmatrix}.
\end{align*}
Note that $\<\bQ^{(1)}\bh^{(0)}_i,\bK^{(1)}\bh^{(0)}_j\>\leq0$ for $i=2t'-1+k'$ (i.e., all tokens that embed the action and reward) since $\oddeven_i=0$, it follows that no update happens on the tokens in which we embed the action and reward (i.e., the corresponding part of $\bh^\partd$ remains zero). Moreover, it should be noted that no  update happens on tokens with $h=1$.

We then use another  attention layer to  multiply the updated vectors by a factor of $2t'+k'$, namely, to perform the map
$$\frac{1}{2t'+k'}\begin{bmatrix}
        \action_{k',h'} \\\reward_{k',h'}\\\state_{k',h'}
    \end{bmatrix}\mapsto\begin{bmatrix}
        \action_{k',h'} \\\reward_{k',h'}\\\state_{k',h'}
    \end{bmatrix},$$ where the output vector is supported on coordinates different from the input vectors. This can be achieved by choosing $\lops{\bQ_1^{(2)}}\leq (2\horizon+1)\Numepi, \lops{\bK_1^{(2)}}\leq (2\horizon+1)\Numepi, \lops{\bV_1^{(2)}}\leq1$
such that \begin{align}
    &
\bQ^{(2)}_{1}\bh^{(1)}_{i}=\begin{bmatrix}
         i^2\\-(2\horizon+1)\Numepi i^2\\  1\\ \bzero
    \end{bmatrix},~~ \bK^{(2)}_{1}\bh^{(1)}_{j}=\begin{bmatrix}
        1\\1\\ (2\horizon+1)\Numepi j^2\\\bzero
    \end{bmatrix},~~
\bV^{(2)}_{1}\bh^{(1)}_{2t'+k'}=\frac{1}{2t'+k'}\begin{bmatrix}
        \bzero\\
        \action_{k',h'} \\\reward_{k',h'}\\\state_{k',h'}
    \\ \bzero
    \end{bmatrix},\label{eq:tf_constrcut_ucbvi_multi}
\end{align}
and noting that $\<\bQ^{(2)}_1\bh_i^{(1)},\bQ^{(2)}_1\bh_j^{(1)}\>=i$ when $j=i$ and otherwise $0$.

For step~\ref{mdp_step1b},  we show that it can be implemented using a two-layer attention-only transformer.

To compute $\Numvi_\sh(\state,\action,\state')$, in the first layer we construct $\head=10\horizon\Numst^2\Numact$ heads with the query, key, value matrices $\{\bQ^{(1)}_{\si\sj\sk\sh,s}\}_{s=1}^{10},\{\bK^{(1)}_{\si\sj\sk\sh,s}\}_{s=1}^{10},\{\bV^{(1)}_{\si\sj\sk\sh,s}\}_{s=1}^{10}$  such that for all $i\leq 2(t-1)+k$ and $\si,\sk\in[\Numst],\sj\in[\Numact],\sh\in[\horizon]$
\begin{align*}
&\bQ^{(1)}_{\si\sj\sk\sh,1}\bh^{(0)}_{i}=
\begin{bmatrix}
\tfthres(\oddeven_i-1)\\
        \tfthres\be_\si\\
         \tfthres\be_\sj\\
          \tfthres\be_\sk\\
          1\\
          1\\
         \sh
    \end{bmatrix},~~ \bK^{(1)}_{\si\sj\sk\sh,1}\bh^{(0)}_{i}=\begin{bmatrix}1\\
     \state_{\bar{k}(i),\bar{h}(i)-1}\\
        \action_{\bar{k}(i),\bar{h}(i)-1}\\
         \state_{\bar{k}(i),\bar{h}(i)}\\
        -3\tfthres\\
        1- \bar{h}(i)\\
        1
\end{bmatrix},~~ \bV^{(1)}_{\si\sj\sk\sh,1}\bh^{(0)}_{i}=-
\begin{bmatrix}
\bzero\\\be^{\Numvi_\sh}_{\si\sj\sk}
\\\bzero\end{bmatrix},
\end{align*}
where we choose $\tfthres=2\horizon$ and $\be_{\si\sj\sk}^\sh$ denotes the one-hot vector supported on the $(\si,\sj,\sk)$-entry in $\Numvi_\sh(\state,\action,\state')$.
We similarly construct
\begin{align*}
&\bQ^{(1)}_{\si\sj\sk\sh,2}\bh^{(0)}_{i}=
\begin{bmatrix}
\tfthres(\oddeven_i-1)\\
        \tfthres\be_\si\\
         \tfthres\be_\sj\\
          \tfthres\be_\sk\\
          1\\
          1\\
         \sh
    \end{bmatrix},~~ \bK^{(1)}_{\si\sj\sk\sh,2}\bh^{(0)}_{i}=\begin{bmatrix}1\\
     \state_{\bar{k}(i),\bar{h}(i)-1}\\
        \action_{\bar{k}(i),\bar{h}(i)-1}\\
         \state_{\bar{k}(i),\bar{h}(i)}\\
        -3\tfthres\\
     - \bar{h}(i)\\
        1
\end{bmatrix},~~ \bV^{(1)}_{\si\sj\sk\sh,2}\bh^{(0)}_{i}=
\begin{bmatrix}
\bzero\\\be^{\Numvi_\sh}_{\si\sj\sk}
\\\bzero\end{bmatrix},
\\
&\bQ^{(1)}_{\si\sj\sk\sh,3}\bh^{(0)}_{i}=
\begin{bmatrix}
\tfthres(\oddeven_i-1)\\
        \tfthres\be_\si\\
         \tfthres\be_\sj\\
          \tfthres\be_\sk\\
          1\\
          1\\
         -\sh
    \end{bmatrix},~~ \bK^{(1)}_{\si\sj\sk\sh,3}\bh^{(0)}_{i}=\begin{bmatrix}1\\
     \state_{\bar{k}(i),\bar{h}(i)-1}\\
        \action_{\bar{k}(i),\bar{h}(i)-1}\\
         \state_{\bar{k}(i),\bar{h}(i)}\\
        -3\tfthres\\
      \bar{h}(i)-1\\
        1
\end{bmatrix},~~ \bV^{(1)}_{\si\sj\sk\sh,3}\bh^{(0)}_{i}=-
\begin{bmatrix}
\bzero\\\be^{\Numvi_\sh}_{\si\sj\sk}
\\\bzero\end{bmatrix},\\
&\bQ^{(1)}_{\si\sj\sk\sh,4}\bh^{(0)}_{i}=
\begin{bmatrix}
\tfthres(\oddeven_i-1)\\
        \tfthres\be_\si\\
         \tfthres\be_\sj\\
          \tfthres\be_\sk\\
          1\\
          1\\
         -\sh
    \end{bmatrix},~~ \bK^{(1)}_{\si\sj\sk\sh,4}\bh^{(0)}_{i}=\begin{bmatrix}1\\
     \state_{\bar{k}(i),\bar{h}(i)-1}\\
        \action_{\bar{k}(i),\bar{h}(i)-1}\\
         \state_{\bar{k}(i),\bar{h}(i)}\\
        -3\tfthres\\
      \bar{h}(i)-2\\
        1
\end{bmatrix},~~ \bV^{(1)}_{\si\sj\sk\sh,4}\bh^{(0)}_{i}=
\begin{bmatrix}
\bzero\\\be^{\Numvi_\sh}_{\si\sj\sk}
\\\bzero\end{bmatrix},\\
&\bQ^{(1)}_{\si\sj\sk\sh,5}\bh^{(0)}_{i}=
\begin{bmatrix}
\tfthres(\oddeven_i-1)\\
        \tfthres\be_\si\\
         \tfthres\be_\sj\\
          \tfthres\be_\sk\\
          1\\
    \end{bmatrix},~~ \bK^{(1)}_{\si\sj\sk\sh,5}\bh^{(0)}_{i}=\begin{bmatrix}1\\
     \state_{\bar{k}(i),\bar{h}(i)-1}\\
        \action_{\bar{k}(i),\bar{h}(i)-1}\\
         \state_{\bar{k}(i),\bar{h}(i)}\\
        -3\tfthres\\
\end{bmatrix},~~ \bV^{(1)}_{\si\sj\sk\sh,5}\bh^{(0)}_{i}=
\begin{bmatrix}
\bzero\\\be^{\Numvi_\sh}_{\si\sj\sk}
\\\bzero\end{bmatrix}.\\
\end{align*}
Summing up the first five heads, we verify that such attention updates the token with $\bh_i^\parta=\bzero$ and has the form
\begin{align*}
\bzero\rightarrow\bzero+\frac1i\widetilde{\Numvi}_\sh(\si,\sj,\sk)\be_{\si\sj\sk}^{\Numvi_h}
\end{align*} on  $\bh^\partd_i$, where $\widetilde{\Numvi}_\sh(\si,\sj,\sk)$ denote the number of visits to the state-action-next-state tuple $(\si,\sj,\sk)$ at time step $\sh$ before token $i$. For $\star\in[5]$, we choose $\bV^{(1)}_{\si\sj\sk\sh,\star+5}=-\bV^{(1)}_{\si\sj\sk\sh,\star+5}$ and $\bQ^{(1)}_{\si\sj\sk\sh,\star+5},\bK^{(1)}_{\si\sj\sk\sh,\star+5}$ be such that
\begin{align*}
&\bQ^{(1)}_{\si\sj\sk\sh,\star+5}\bh^{(0)}_{i}=
\begin{bmatrix}
\bQ^{(1)}_{\si\sj\sk\sh,\star}\bh^{(0)}_{i}\\
\tfthres\\
-\bar{k}(i)
    \end{bmatrix},~~ \bK^{(1)}_{\si\sj\sk\sh,\star+5}\bh^{(0)}_{i}=\begin{bmatrix}
\bK^{(1)}_{\si\sj\sk\sh,\star}\bh^{(0)}_{i}\\ \bar{k}(i)\\\tfthres
\end{bmatrix}
\end{align*} which adds positional embedding about the current episode $\bar{k}(i)$. We verify that summing up the sixth to the tenth heads gives the update  \begin{align*}
\bzero\rightarrow\bzero+\frac{1}{i}({\Numvi}_\sh(\si,\sj,\sk)-\widetilde{\Numvi}_\sh(\si,\sj,\sk))\be_{\si\sj\sk}^{\Numvi_h}
\end{align*} on  $\bh^\partd_i$ for $i\leq 2(t-1)+k$ with $\bh_i^\parta=\bzero$. Therefore, combining all the heads together we have the update
\begin{align*}
\bzero\rightarrow\bzero+\frac{1}{i}{\Numvi}_\sh(\si,\sj,\sk)\be_{\si\sj\sk}^{\Numvi_h}\text{~~for all~}\si,\sk\in[\Numst],\sj\in[\Numact],\sh\in[\horizon]
\end{align*} on $\bh^\partd_i$ for $i\leq 2(t-1)+k$ with $\bh_i^\parta=\bzero$, in particular when $i=2(t-1)+k$. Moreover,  notice that the matrices $\{\bQ^{(1)}_{\si\sj\sk\sh,s}\}_{s=1}^{10},\{\bK^{(1)}_{\si\sj\sk\sh,s}\}_{s=1}^{10}$ can be constructed with the operator norm less than $10\tfthres=10\horizon$, and $\{\bV^{(1)}_{\si\sj\sk\sh,s}\}_{s=1}^{10}$ with the operator norm equals $1$.

Following a similar construction, we can also compute $\Numvi_\sh(\state,\action),\Numvi_\sh(\state,\action)\reward_\sh(\state,\action)$ for all $\sh,\state,\action,\state'$ on different supports of coordinates in $\bh_i^\partd$ via adding additional $\head=\conO(\horizon\Numst\Numact)$ heads to the attention-only layer.


Next, we construct the second attention layer to multiply the token vector by the index number $i$ as in the proof of Step~\ref{mdp_step1a}. The construction is similar to that in Eq.~\eqref{eq:tf_constrcut_ucbvi_multi} and we omit it here. Moreover, note that Step~\ref{mdp_step1b} can be implemented with the embedding dimension $D\leq\conO(\horizon\Numst^2\Numact)$ as we need $\conO(1)$ dimensions for each quadruple $(\si,\sj,\sk,\sh)$.  Combining Step~\ref{mdp_step1a},~\ref{mdp_step1b} concludes the proof of Step~\ref{mdp_step1}.










\paragraph{Proof of Step~\ref{mdp_step2}}
After Step~\ref{mdp_step1}, for the current token $i=2(t-1)+k$, we have $\Numvi_\sh(\state,\action,\state'),\reward_\sh(\state,\action),\Numvi_\sh(\state,\action)$, $\Numvi_\sh(\state,\action)\reward_\sh(\state,\action)$  lie in $\bh_i^\partd$ for all $\sh\in[\horizon]$. Given these vectors that store the number of visits and rewards, note that
\begin{align*}
\reward_\sh(\state,\action)&=\frac{\Numvi_\sh(\state,\action)\reward_\sh(\state,\action)}{\Numvi_\sh(\state,\action)\vee1},~~\text{ when  } \Numvi_\sh(\state,\action)\geq1,~~~\\
\bonus_\sh(\state,\action)
&=2\horizon\sqrt{\frac{\log(\Numst\Numact\totlen/\delta)}{\Numvi_\sh(\state,\action)\vee1}},
\\
\tresttransit_\sh(\state,\action,\state')
&=\frac{\Numvi_\sh(\state,\action,\state')}{\Numvi_\sh(\state,\action)\vee 1}.
\end{align*}

Therefore, we may compute $\esttransit_\sh,\estbonus_\sh$ via using a transformer layer  to implement the functions $f_1(x,y)=\frac{x}{y\vee1},f_2(y)=2\horizon\sqrt{\frac{\log(\Numst\Numact\totlen/\delta)}{y\vee 1}},f_3(x,y)=\frac{x}{y\vee1}+\horizon\bone_{y=0}$ for $x,y\in\{0\}\cup[\Numepi]$.
We demonstrate the computation of $\esttransit_\sh(\state,\action,\state')$  (i.e., the computation of $f_1(x,y)$) here. We start with constructing an attention layer with $\head=\conO(\horizon\Numst^2\Numact)$ heads such that it implements  $x\mapsto x^2$ for $x=\Numvi_\sh(\state,\action,\state'),\Numvi_\sh(\state,\action)$. For $\Numvi_\sh(\state,\action,\state')$, this can be done by  choosing  $\lops{\bQ_{\si\sj\sk\sh}^{(1)}}\leq\Numepi,\lops{\bK_{\si\sj\sk\sh}^{(1)}}\leq\Numepi,\lops{\bV_{\si\sj\sk\sh}^{(1)}}=1$ such that
\begin{align*}
    \bQ_{\si\sj\sk\sh}^{(1)}\bh_i^{(0)}=\begin{bmatrix}
        \Numepi\\
        -i\\
        \Numvi_\sh(\be_\si,\be_\sj,\be_\sk)
    \end{bmatrix},~~\bK_{\si\sj\sk\sh}^{(1)}\bh_j^{(0)}=\begin{bmatrix}
        j\\ \Numepi\\
        \Numvi_\sh(\be_\si,\be_\sj,\be_\sk)
    \end{bmatrix},~~
    \bV_{\si\sj\sk\sh}^{(1)}\bh_j^{(0)}=\begin{bmatrix}
        \bzero\\j\\\bzero
    \end{bmatrix},
\end{align*} where $\be_\si,\be_\sk$ denote the $i,j$-th states and $\be_{sj}$ denotes the $k$-th action.
Similarly, we can construct $\horizon\Numst\Numact$ additional heads
to compute $\Numvi_\sh(\state,\action)^2$ for all possible $\state,\action.$

Next, we compute the exact values of $\esttransit(\state,\action,\state')$ using an MLP layer. Namely,
we  construct $\bW^{(1)}_1=\bW^{(1)}_{12}\bW^{(1)}_{11},\bW^{(1)}_2=\bW^{(1)}_{23}\bW^{(1)}_{22}\bW^{(1)}_{21}$  such that for all $\sh,\state,\action,\state'$, on the corresponding vector component we have
\begin{align*}
   & \bW^{(1)}_{11}\bh_i^{(0)}=     \begin{bmatrix}
    1\\
       \Numvi_\sh(\state,\action,\state')^2 \\\vdots
       \\
        (\Numvi_\sh(\state,\action,\state')-\Numepi)^2\\
         \Numvi_\sh(\state,\action)^2 \\\vdots
       \\
        (\Numvi_\sh(\state,\action)-\Numepi)^2
    \end{bmatrix}= \begin{bmatrix}
    1\\
       \Numvi_\sh(\state,\action,\state')^2 \\\vdots
       \\
        \Numvi_\sh(\state,\action,\state')^2+\Numepi^2-2\Numepi\Numvi_\sh(\state,\action,\state')\\
          \Numvi_\sh(\state,\action)^2 \\\vdots
       \\
        \Numvi_\sh(\state,\action)^2+\Numepi^2-2\Numepi\Numvi_\sh(\state,\action)
    \end{bmatrix},
    \\
&\bW^{(1)}_{12}\bW^{(1)}_{11}\bh_i^{(0)}=     \begin{bmatrix}
    1-
       \Numvi_\sh(\state,\action,\state')^2 - \Numvi_\sh(\state,\action)^2 \\\vdots
       \\ 1-
       (\Numvi_\sh(\state,\action,\state')-x)^2 - (\Numvi_\sh(\state,\action)-y)^2 \\\vdots\\
       1- (\Numvi_\sh(\state,\action,\state')-\Numepi)^2-(\Numvi_\sh(\state,\action)-\Numepi)^2
    \end{bmatrix} ,
\end{align*} where $x,y\in\{0\}\cup[\Numepi]$. Moreover,  we construct $\bW^{(1)}_{21}$ so that on the entries corresponding to $\sh,\state,\action,\state'$ it implements
\begin{align*}
\bW_{2}^{(1)}\sigma(\bW_{1}^{(1)}\bh_i^{(0)})
    =\begin{bmatrix}
        \sum_{x,y=0}^{\Numepi}\sigma(1- (\Numvi_\sh(\state,\action,\state')-x)^2 - (\Numvi_\sh(\state,\action)-y)^2)\cdot\frac{x}{y\vee1}
    \end{bmatrix}=\begin{bmatrix}
        \frac{\Numvi_\sh(\state,\action,\state')}{\Numvi_\sh(\state,\action)\vee1}.
    \end{bmatrix}
\end{align*}

It can be verified that we can find such $\bW_1^{(1)},\bW_2^{(1)}$ with $$\lops{\bW_1^{(1)}}\leq \lops{\bW_{11}^{(1)}}\lops{\bW_{12}^{(1)}}\leq \conO(\Numepi^2)\cdot \conO(\Numepi)=\conO(\Numepi^3),$$ $\lops{\bW_2^{(1)}}\leq \conO(\Numepi)
$, and the number of hidden neurons $\hidden=\conO(\Numepi^2\horizon\Numst^2\Numact)$. Simlarly, we can compute $f_2(\cdot)$ (or $f_3(\cdot)$) exactly following the same construction but with a different $\bW_2^{(1)}$ that records all possible values of  $f_2(\cdot)$ (or $f_3(\cdot)$). Combining the upper bounds on the operator norm of the  weight matrices, we further have $\nrmp{\tfpar}\leq\tcO(\horizon\Numst^2\Numact+\Numepi^3+\Numepi\horizon)$.















\paragraph{Proof of Step~\ref{mdp_step3}} Given $\trestVfun_{\horizon+1}=\estVfun_{\horizon+1}=\bzero$, we show the there exists an transformer with
\begin{align*}
\layer=2,~~~\max_{\ell\in[\layer]}\head^\lth\leq2\Numst\Numact,~~~\max_{\ell\in[\layer]}\hidden^{\lth}\leq3\Numst\Numact,~~~\nrmp{\tfpar}\leq \conO(\horizon+\Numst\Numact)
\end{align*}
that implements one step of value iteration
\begin{align*}
    \estQfun_\sh(\state,\action)&=\max\{\min\{\horizon,\estreward_\sh(\state,\action)+\estbonus_\sh(\state,\action)+\sum_{\state'\in\statesp}\esttransit_\sh(\state'\mid\state,\action)\estVfun_{\sh+1}(\state')\},0\},\\
\estVfun_\sh(\state)&=\max_{\action\in\actionsp}\estQfun_\sh(\state,\action)
\end{align*} for some $\sh\in[\horizon]$.
Namely, we start with constructing an-attention layer with $\head=2\Numst\Numact$ and $\{\bQ_{\si\sj\sh,s}^{(1)}\}^2_{s=1}$, $\{\bK_{\si\sj\sh,s}^{(1)}\}^2_{s=1}$, $\{\bV_{\si\sj\sh,s}^{(1)}\}^2_{s=1}$ such that for all $i\leq 2(t-1)+k$
\begin{align*}
   \bQ_{\si\sj,1}^{(1)}\bh_i^{(0)} =\begin{bmatrix}
       \tfthres\\-i\\
       \estVfun_{\sh+1}(\cdot)\\
   \end{bmatrix},~~ \bK_{\si\sj,1}^{(1)}\bh_i^{(0)} =\begin{bmatrix}
       i\\ \tfthres\\
       \esttransit_{\sh+1}(\cdot|\state,\action)\\
   \end{bmatrix},~~~
   \bV_{\si\sj,1}^{(1)}\bh_i^{(0)} =\begin{bmatrix}
       \bzero \\ i\be^{\Qfun_\sh}_{\si\sj}\\
       \bzero,
   \end{bmatrix}
   \\
    \bQ_{\si\sj,2}^{(1)}\bh_i^{(0)} =\begin{bmatrix}
       \tfthres\\-i\\
       -\estVfun_{\sh+1}(\cdot)\\
   \end{bmatrix},~~ \bK_{\si\sj,2}^{(1)}=\bK_{\si\sj,1}^{(1)},~~~
\bV_{\si\sj,2}^{(1)}=-\bV_{\si\sj,2}^{(1)}
\end{align*}
where $\tfthres=3\horizon$ and $\be^{\Qfun_\sh}_{\si\sj}\in\R^{\Numst\Numact}$ is a vector supported on some coordinates in $\bh^\partd_i$ reserved for $\estQfun_\sh$. Moreover, we have $\lops{\bQ^{(1)}_{\si\sj\sh,s}},\lops{\bK^{(1)}_{\si\sj\sh,s}}\leq \tfthres,~\lops{\bV^{(1)}_{\si\sj\sh,s}}=1$.
Since $$\Big|\<\estVfun_{\sh+1}(\cdot),\esttransit_{\sh+1}(\cdot|\state,\action)\>\Big|\leq \linf{\estVfun_{\sh+1}(\cdot)}\cdot\lone{\esttransit_{\sh+1}(\cdot|\state,\action)}\leq\horizon$$ as $\estVfun_{\sh+1}(\state)\in[0,\horizon]$ and $\lone{\esttransit_{\sh+1}(\cdot|\state,\action)}=1$, it follows that summing up two heads gives the update for $i\leq 2(t-1)+k$
\begin{align*}
\bzero
&\mapsto\bzero+\Big[\sigma(\< \bQ_{\si\sj,1}^{(1)}\bh_i^{(0)} , \bK_{\si\sj,1}^{(1)}\bh_j^{(0)} \>)-\sigma(\< \bQ_{\si\sj,1}^{(1)}\bh_i^{(0)} , \bK_{\si\sj,1}^{(1)}\bh_j^{(0)} \>)\Big]\be_{\si\sj}^{\Qfun_\sh}\\
&=\< \bQ_{\si\sj,1}^{(1)}\bh_i^{(0)} , \bK_{\si\sj,1}^{(1)}\bh_j^{(0)} \>\be_{\si\sj}^{\Qfun_\sh}.
\end{align*}
Denote the resulting token vector by $\bh_i^{(1)}$.
Moreover, we can construct a two-layer MLP with $$\lops{\bW^{(1)}_1}=\conO(\horizon),~~ \lops{\bW^{(1)}_2}\leq 3,~~\hidden=3\Numst\Numact$$
such that for any state-action pair $(\state,\action)\in\statesp\times\actionsp$ on the corresponding coordinates
\begin{align*}
\bW^{(1)}_1\bh_i^{(1)}=
\begin{bmatrix}
\vdots\\-[\estreward_\sh(\state,\action)+\estbonus_\sh(\state,\action)+\sum_{\state'\in\statesp}\esttransit_\sh(\state'\mid\state,\action)\estVfun_{\sh+1}(\state')]
\\
\estreward_\sh(\state,\action)+\estbonus_\sh(\state,\action)+\sum_{\state'\in\statesp}\esttransit_\sh(\state'\mid\state,\action)\estVfun_{\sh+1}(\state')-\horizon\\\estreward_\sh(\state,\action)+\estbonus_\sh(\state,\action)+\sum_{\state'\in\statesp}\esttransit_\sh(\state'\mid\state,\action)\estVfun_{\sh+1}(\state')\\\vdots
\end{bmatrix}
\end{align*}
 and \begin{align*}
  \bW^{(1)}_2\sigma(\bW^{(1)}_1\bh_i^{(1)})
  &=   \sigma(-[\estreward_\sh(\state,\action)+\estbonus_\sh(\state,\action)+\sum_{\state'\in\statesp}\esttransit_\sh(\state'\mid\state,\action)\estVfun_{\sh+1}(\state')])\\&\quad-\sigma(\estreward_\sh(\state,\action)+\estbonus_\sh(\state,\action)+\sum_{\state'\in\statesp}\esttransit_\sh(\state'\mid\state,\action)\estVfun_{\sh+1}(\state')-\horizon)\\
&\qquad+\sigma(\estreward_\sh(\state,\action)+\estbonus_\sh(\state,\action)+\sum_{\state'\in\statesp}\esttransit_\sh(\state'\mid\state,\action)\estVfun_{\sh+1}(\state'))\\
&=\max\{\min\{\horizon,\estreward_\sh(\state,\action)+\estbonus_\sh(\state,\action)+\sum_{\state'\in\statesp}\esttransit_\sh(\state'\mid\state,\action)\estVfun_{\sh+1}(\state')\},0\}=\estQfun_h(\state,\action).
 \end{align*}
 Denote the resulting token vector by $\bh_i^{(2)}$.
 Next, we construct a second MLP layer with $$\lops{\bW_1^{(2)}}\leq2,~~\lops{\bW_2^{(2)}}\leq\sqrt{\Numact},~~\hidden=\Numact\Numst
 $$ such that for any $\state\in\statesp$ on the corresponding coordinates we have
 \begin{align*}
     \bW_1^{(2)}\bh_i^{(2)}=\begin{bmatrix}
        \vdots\\ \estQfun_\sh(\state,\action_1)\\
         \estQfun_\sh(\state,\action_2)-\estQfun_\sh(\state,\action_1)\\
         \vdots\\
          \estQfun_\sh(\state,\action_\Numact)-\estQfun_\sh(\state,\action_{\Numact-1})
    \\ \vdots\end{bmatrix},
 \end{align*}
 where $\action_j$ denotes the $j-$th action,
 and
 \begin{align*}
  \bW^{(2)}_2\sigma(\bW^{(2)}_1\bh_i^{(2)})
 &=
 \sigma(\estQfun_\sh(\state,\action_1))+\sum_{j=2}^\Numact\sigma( \estQfun_\sh(\state,\action_j)-\estQfun_\sh(\state,\action_{j-1}))\\
 &=\max_{\action\in\actionsp}\estQfun_\sh(\state,\action)=\estVfun_\sh(\state).
 \end{align*}
 Using the upper bounds on the operator norm of the  weight matrices, we further have $\nrmp{\tfpar}\leq\conO(\Numst\Numact+\horizon)$.
Combining the steps concludes the construction in Step~\ref{mdp_step3}.





 \paragraph{Proof of Step~\ref{mdp_step4}}
 we start with constructing an-attention layer with $\head=2\horizon\Numact$ and $\{\bQ_{\sj\sh,s}^{(1)}\}^2_{s=1}$, $\{\bK_{\sj\sh,s}^{(1)}\}^2_{s=1}$, $\{\bV_{\sj\sh,s}^{(1)}\}^2_{s=1}$ such that for all the current token $i= 2(t-1)+k$ and $j\leq i$
\begin{align*}
  & \bQ_{\sj\sh,1}^{(1)}\bh_i^{(0)} =\begin{bmatrix}
       \state_{\bar{k}(i),\bar{h}(i)}\\-i\\
       \tfthres
   \end{bmatrix},~~ \bK_{\sj\sh,1}^{(1)}\bh_j^{(0)} =\begin{bmatrix}
    \estQfun_\sh(\cdot,\action_\sj)\\  \tfthres\\ j
   \end{bmatrix},~~~
   \bV_{\sj\sh,1}^{(1)}\bh_i^{(0)} =\begin{bmatrix}
       \bzero \\ i\be_{\sj\sh}\\
       \bzero,
   \end{bmatrix}
   \\
    &\bQ_{\sj\sh,2}^{(1)}\bh_i^{(0)} =\begin{bmatrix}
       -\state_{\bar{k}(i),\bar{h}(i)}\\-i\\
       \tfthres
   \end{bmatrix},~~ \bK_{\sj\sh,2}^{(1)}=\bK_{\sj\sh,1}^{(1)},~~~
\bV_{\sj\sh,2}^{(1)}=-\bV_{\sj\sh,1}^{(1)},
\end{align*}
where we choose $\tfthres=2\horizon$ and $\bV_{\sj\sh,1}^{(1)}\bh_i^{(0)}$ is a one-hot vector supported on some entry of $\bh_i^\partd$. We verify  that summing up the heads gives the update
\begin{align*}
    \bzero\mapsto \estQfun_{\sh}(\state_{k,h},\action_\sj)\be_{\sj\sh}
\end{align*}
 for all $\sh\in[\horizon],\sj\in[\Numact]$. Moreover, we have $\lops{\bQ_{\sj\sh,s}^{(1)}}\leq2\horizon,\lops{\bK_{\sj\sh,s}^{(1)}}\leq2\horizon,\lops{\bV_{\sj\sh,s}^{(1)}}\leq1$ for $s=1,2$. Through this attention-only layer, we extract the values $\estQfun_\sh(\state_{k,h},\action_j)$ for all $\sh\in[\horizon]$ from the Q-function.

Similar to the proof of Step~\ref{mdp_step1b}, we construct a second attention-only layer with attention heads   $\{\bQ_{\sj\sh,s}^{(2)}\}^2_{s=1}$, $\{\bK_{\sj\sh,s}^{(2)}\}^2_{s=1}$, $\{\bV_{\sj\sh,s}^{(2)}\}^2_{s=1}$ that
\begin{align*}
  & \bQ_{\sj\sh,1}^{(2)}\bh_i^{(1)} =\begin{bmatrix}
       1\\-\bar{h}(i)\\-i\\
       \tfthres
   \end{bmatrix},~~ \bK_{\sj\sh,1}^{(2)}\bh_j^{(1)} =\begin{bmatrix}
   \sh\\ 1 \\\tfthres\\ j
   \end{bmatrix},~~~
   \bV_{\sj\sh,1}^{(2)}\bh_i^{(1)} =-\begin{bmatrix}
       \bzero \\  \estQfun_\sh(\state_{k,h},\action_\sj)\be_{\sj}\\
       \bzero
   \end{bmatrix},
   \\
   & \bQ_{\sj\sh,2}^{(2)}=\bQ_{\sj\sh,1}^{(2)},~~ \bK_{\sj\sh,2}^{(2)}\bh_j^{(1)} =\begin{bmatrix}
   \sh-1\\ 1 \\\tfthres\\ j
   \end{bmatrix},~~~
   \bV_{\sj\sh,2}^{(2)} =-  \bV_{\sj\sh,2}^{(2)},
\end{align*} where $ \bV_{\sj\sh,1}^{(2)}\bh_i^{(1)}$ are supported on some entry of $\bh_i^\partd$ for $s=1,2$.
Summing up the heads gives the update
\begin{align*}
    \bzero\mapsto -\sum_{s=\bar{h}(i)+1}^{\horizon}\frac{1}{i}\estQfun_{s}(\state_{k,h},\action_\sj).
\end{align*}
Similarly, we can construct attention heads   $\{\bQ_{\sj\sh,s}^{(2)}\}^4_{s=3},\{\bK_{\sj\sh,s}^{(2)}\}^4_{s=3},\{\bV_{\sj\sh,s}^{(2)}\}^4_{s=3}$ that implements
\begin{align*}
    \bzero\mapsto -\frac{1}{i}\sum_{s=1}^{\bar{h}(i)-1}{\estQfun_{s}(\state_{k,h},\action_\sj)}.
\end{align*}
Moreover, we construct
$\bQ_{\sj\sh,5}^{(2)},\bK_{\sj\sh,5}^{(2)},\bV_{\sj\sh,5}^{(2)}$ with
\begin{align*}
  & \bQ_{\sj\sh,5}^{(2)}\bh_i^{(1)} =\begin{bmatrix}
       1\\-i\\
       \tfthres
   \end{bmatrix},~~ \bK_{\sj\sh,5}^{(2)}\bh_j^{(1)} =\begin{bmatrix}
    1 \\\tfthres\\ j
   \end{bmatrix},~~~
   \bV_{\sj\sh,1}^{(2)}\bh_i^{(1)} =\begin{bmatrix}
       \bzero \\  \estQfun_\sh(\state_{k,h},\action_\sj)\be_{\sj}\\
       \bzero,
   \end{bmatrix}
\end{align*} that implements
\begin{align*}
    \bzero\mapsto \frac{1}{i}\sum_{s=1}^{\horizon}{\estQfun_{s}(\state_{k,h},\action_\sj)}.
\end{align*}
Therefore, summing up the $\head=5\horizon\Numact$ heads we obtain the update
\begin{align*}
  \bzero_{\Numact}\mapsto \frac{1}{i}\estQfun_{h}(\state_{k,h},\cdot).
\end{align*}
Note that $\lops{\bQ_{\sj\sh,s}^{(1)}}\leq4\horizon,\lops{\bK_{\sj\sh,s}^{(1)}}\leq4\horizon,\lops{\bV_{\sj\sh,s}^{(1)}}\leq1$ for $s\in[5]$.

Finally, we apply an attention-only layer to implement the multiplication by a factor of $i/\temp$ using a similar construction as in Eq.~\eqref{eq:tf_constrcut_ucbvi_multi} with $\lops{\bQ_1^{(3)}}=\conO(\horizon\Numepi),\lops{\bK_1^{(3)}}=\conO(\horizon\Numepi),\lops{\bV_1^{(3)}}=\conO(1/\temp)$, and assign the resulting vector $\estQfun(\state_{k,h},\cdot)/\temp$ to $\bh_i^\partc$.
Combining the three attention-only layers completes Step~\ref{mdp_step4}.





















\subsection{Proof of Theorem~\ref{thm:ucbvi_icrl-main}}\label{sec:pf_thm:ucbvi_icrl-main}
By Theorem~\ref{thm:diff_reward}~and~\ref{thm:approx_ucbvi}, it suffices to show the regret of soft UCB-VI satisfies
\begin{align*}
\E[\Numepi\Vfun_\inst(\plc^*)-\totreward_{\inst,\sAlg_\sUCBVI(\temp)}(\totlen)]\leq \tcO (\horizon^2\sqrt{\Numst\Numact\Numepi}+\horizon^3\Numst^2\Numact)
\end{align*} for all MDP instances $\inst$, where $\temp=1/\Numepi$ and $\tcO(\cdot)$ hides logarithmic dependencies on $(\horizon,\Numepi,\Numst,\Numact)$.


Throughout the proof, we may drop the dependence on $\inst$ for notational simplicity when there is no confusion. For each episode $k\in[\Numepi]$, let $\Numvi^k_\sh,\tresttransit^k_\sh,\trestQfun^k_\sh,\trestVfun^k_\sh,\bonus^k_\sh$ denote the corresponding quantities $\Numvi_\sh,\tresttransit_\sh,\trestQfun_\sh,\trestVfun_\sh,\bonus_\sh$  introduced in UCB-VI (see Section~\ref{sec:example_ucbvi}).

For a policy $\plc$ and time step $\sh\in[\horizon]$, we define the Q-function $\Qfun_\sh^\plc$ and the value function $\Vfun_\sh^\plc$
\begin{align*}
\Qfun_\sh^\plc(\state,\action)&:=\E[\sum_{t=\sh}^\horizon \reward(\state_t,\action_t)\mid \state_\sh=\state,\action_\sh=\action,\plc],\\
\Vfun_\sh^\plc(\state)&:=\E[\sum_{t=\sh}^\horizon \reward(\state_t,\action)\mid \state_\sh=\state,\plc].
\end{align*}
 We use $\plc^{k}=(\plc^k_1,\ldots,\plc^k_\horizon),\plc^k_{\s}=(\plc^k_{\s,1},\ldots,\plc^k_{\s,\sh},\ldots\plc^k_{\s,\horizon})$ to denote the policies given by UCB-VI and soft UCB-VI in the $k$-th episode, respectively. Note that we have $\Vfun_\inst(\plc)=\E_{\state\sim\init}[\Vfun_\sh^\plc(\state)]$ and cumulative the regret
 $$
 \E[\Numepi\Vfun_\inst(\plc^*)-\totreward_{\inst,\sAlg_\sUCBVI(\temp)}(\totlen)]]=\E\Big[\sum_{k=1}^{\Numepi}[\Vfun_1^{\optplc}(\state_{k,1})-\Vfun_1^{\plc^k_{\s}}(\state_{k,1})
 ]\Big]
 $$
 where the expectation is taken over the collected data $$\dset_\totlen=\{(\state_{k,\sh},\action_{k,\sh},\reward_{k,\sh})\}_{k\in[\Numepi],\sh\in[\horizon]}\sim\P_\inst^{\sUCBVI(\temp)}.$$
 For any function $f=f(\state,\action)$, we abuse the notation $f(\state,\plc(
\cdot)):=\E_{\action\sim\plc}[f(\state,\action)]$. Lastly, we define
\begin{align*}
\epstemp=\max_{k\in[\Numepi],\sh\in[\horizon],\state\in\statesp}[\trestQfun^k_\sh(\state,\plc^k_{\sh}(\cdot))-\trestQfun^k_\sh(\state,\plc^k_{\s,\sh}(\cdot)) ].
\end{align*}
We claim the following which we will prove later
\begin{align}\label{eq:epstemp_control}
  \epstemp\leq \Numact\temp.
\end{align}

The proof follows from similar arguments as in the proof of Theorem 1 in~\cite{azar2017minimax} (see also Theorem 7.6 in~\cite{agarwal2019reinforcement}). Hence we only provide a sketch of proof here.
First, from the proof of Theorem 7.6 in~\cite{agarwal2019reinforcement} , it can be shown that
\begin{align*}
 \trestVfun_\sh^k(\state)\geq\Vfun^{\optplc}_\sh(\state)
\end{align*}
for any $k,\sh,\state$ with probability at least $1-\delta$.
Thus with probability at least $1-\delta$ for all $\sh\in[\horizon],k\in[\Numepi]$
\begin{align*}
&\qquad\Vfun_\sh^{\optplc}(\state_{k,\sh})- \Vfun^{\plc^k_\s}_\sh(\state_{k,\sh})  \\
&\leq
\trestVfun^k_\sh(\state_{k,\sh})- \Vfun^{\plc^k_\s}_\sh(\state_{k,\sh})  \\
&=\trestQfun^k_\sh(\state_{k,\sh},\plc^k_{\sh}(\cdot))-\trestQfun^k_\sh(\state_{k,\sh},\plc^k_{\s,\sh}(\cdot))+\trestQfun^k_\sh(\state_{k,\sh},\plc^k_{\s,\sh}(\cdot))-
\Qfun^{\plc^k_\s}_\sh(\state_{k,\sh},\plc^k_{\s,\sh}(\cdot))\\
&\leq
\trestQfun^k_\sh(\state_{k,\sh},\plc^k_{\s,\sh}(\cdot))-
\Qfun^{\plc^k_\s}_\sh(\state_{k,\sh},\plc^k_{\s,\sh}(\cdot))+\epstemp\\
&=
\trestQfun^k_\sh(\state_{k,\sh},\action_{k,\sh})-
\Qfun^{\plc^k_\s}_\sh(\state_{k,\sh},\action_{k,\sh})+\MD^{(1)}_{k,\sh}+\epstemp,
\end{align*}
where the first equality uses $\trestVfun^k_\sh(\state_{k,\sh})=\argmax_{\action}\trestQfun^k_\sh(\state_{k,\sh},\action_{k,\sh})=\trestQfun^k_\sh(\state_{k,\sh},\plc_\sh^k(\cdot))$, and in the last line $$
\MD^{(1)}_{k,\sh}:=[\trestQfun^k_\sh(\state_{k,\sh},\plc^k_{\s,\sh}(\cdot))-
\Qfun^{\plc^k_\s}_\sh(\state_{k,\sh},\plc^k_{\s,\sh}(\cdot))]-[\trestQfun^k_\sh(\state_{k,\sh},\action_{k,\sh})-
\Qfun^{\plc^k_\s}_\sh(\state_{k,\sh},\action_{k,\sh})].
$$  Note that for any fixed $\sh\in[\horizon]$, $\{\MD^{(1)}_{k,\sh}\}_{k=1}^{\Numepi}$ is a bounded martingale difference sequence. Following  the proof of Theorem 7.6 in~\cite{agarwal2019reinforcement}, we further have
\begin{align*}
&\quad\Vfun_\sh^{\optplc}(\state_{k,\sh})- \Vfun^{\plc^k_\s}_\sh(\state_{k,\sh})\\
&\leq \trestQfun^k_\sh(\state_{k,\sh},\action_{k,\sh})-
\Qfun^{\plc^k_\s}_\sh(\state_{k,\sh},\action_{k,\sh})+
\MD^{(1)}_{k,\sh}+\epstemp\\
&\leq
\Big(1+\frac{1}\horizon\Big)\Big[\trestVfun_{\sh+1}^k(\state_{k,\sh+1})- \Vfun^{\plc^k_\s}_{\sh+1}(\state_{k,\sh+1})  \Big]+2\bonus_\sh^k(\state_{k,\sh},\action_{k,\sh})
\\
&\qquad+
\frac{c_0L_0\horizon^2\Numst}{\Numvi^k_\sh(\state_{k,\sh},\action_{k,\sh})}+\MD^{(2)}_{k,\sh}+\MD^{(1)}_{k,\sh}+\epstemp,
\end{align*}with probability at least $1-c\delta$ for some universal constant $c>0$,
where $L_0=\log(\Numst\Numact\Numepi\horizon/\delta)$,  $c_0>0$ is some universal constant and
\begin{align*}
 \MD^{(2)}_{k,\sh}:=\transit_\sh(\cdot\mid\state_{k,\sh},\action_{k,\sh})\cdot(\Vfun_{\sh+1}^\optplc-\Vfun_{\sh+1}^{\plc^k_\s})-  (\Vfun_{\sh+1}^\optplc(\state_{k,\sh+  1})-\Vfun_{\sh+1}^{\plc^k_\s}(\state_{k,\sh+  1}))
\end{align*} is a bounded martingale difference sequence for any fixed $\sh\in[\horizon]$. Using the recursive formula and the fact that $(1+1/\horizon)^\horizon<e$, we obtain
\begin{align*}
 &\quad \E\Big[ \sum_{k=1}^\Numepi [\Vfun_1^{\optplc}(\state_{k,1})-\Vfun_1^{\plc^k_{\s}}(\state_{k,1})
 ]  \Big]\\
 &\leq c\E\Bigg[\sum_{k=1}^\Numepi\sum_{\sh=1}^\horizon\Big[2\bonus_\sh^k(\state_{k,\sh},\action_{k,\sh})
+
\frac{c_0L_0\horizon^2\Numst}{\Numvi^k_\sh(\state_{k,\sh},\action_{k,\sh})}+\MD^{(2)}_{k,\sh}+\MD^{(1)}_{k,\sh}\Big]\Bigg]+\E[\Numepi\sum_{\sh=0}^{\horizon-1}(1+\frac{1}\horizon)^\sh\epstemp]\\
&\leq c\E\Bigg[\sum_{k=1}^\Numepi\sum_{\sh=1}^\horizon\Big[2\bonus_\sh^k(\state_{k,\sh},\action_{k,\sh})
+
\frac{c_0L_0\horizon^2\Numst}{\Numvi^k_\sh(\state_{k,\sh},\action_{k,\sh})}+\MD^{(2)}_{k,\sh}+\MD^{(1)}_{k,\sh}\Big]\Bigg]+c\Numepi\horizon\Numact\temp\\
&\leq
\tcO (\horizon^2\sqrt{\Numst\Numact\Numepi}+\horizon^3\Numst^2\Numact)+c\Numepi\horizon\Numact\temp\\
&\leq \tcO(\horizon^2\sqrt{\Numst\Numact\Numepi}+\horizon^3\Numst^2\Numact),
\end{align*} where $c>0$ is some universal constant,  $\tcO(\cdot)$ hides logarithmic dependencies on $(\horizon,\Numepi,\Numst,\Numact)$, and the last line follows again from the proof of Theorem~7.6 in~\cite{agarwal2019reinforcement}, and the assumption that $\temp=1/\Numepi$.
 We omit the detailed derivations here as they are similar to those in~\cite{azar2017minimax,agarwal2019reinforcement}. Therefore, we conclude the proof of the first part of Theorem~\ref{thm:ucbvi_icrl-main}. Moreover,  the second part of Theorem~\ref{thm:ucbvi_icrl-main} (i.e., the upper bound on $\log\cN_{\tfparspace}$) follows immediately from Lemma~\ref{lm:cover_num_bound} and Eq.~\eqref{eq:ucbvi_tf_param-main}.


 \paragraph{Proof of Eq.~\eqref{eq:epstemp_control}}
 By definition of $\trestQfun_\sh^k$ and $\plc_\sh^k,\plc^k_{\s,\sh}$, we have
 \begin{align*}
     \trestQfun^k_\sh(\state,\plc^k_{\sh}(\cdot))-\trestQfun^k_\sh(\state,\plc^k_{\s,\sh}(\cdot)) &=\max_{\action} \trestQfun^k_\sh(\state,\action)-
\sum_{\action}\frac{\exp(\trestQfun^k_\sh(\state,\action)/\temp)}{\sum_\action \exp(\trestQfun^k_\sh(\state,\action)/\temp)}\cdot\trestQfun^k_\sh(\state,\action)\\
&=
\sum_{\action}\frac{\exp(\trestQfun^k_\sh(\state,\action)/\temp)}{\sum_\action \exp(\trestQfun^k_\sh(\state,\action)/\temp)}\cdot[\max_{\action} \trestQfun^k_\sh(\state,\action)-\trestQfun^k_\sh(\state,\action)]\\
&\leq
\sum_{\action}\frac{\exp(\trestQfun^k_\sh(\state,\action)/\temp)}{ \exp(\max_\action\trestQfun^k_\sh(\state,\action)/\temp)}\cdot[\max_{\action} \trestQfun^k_\sh(\state,\action)-\trestQfun^k_\sh(\state,\action)]\\
&\leq \Numact \cdot [\sup_{t\geq 0}t\exp(-t/\temp)]\leq\Numact\temp.
 \end{align*}
